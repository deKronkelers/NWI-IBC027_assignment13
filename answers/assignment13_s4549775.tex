\documentclass[12pt, a4paper]{article}

\usepackage{amssymb}
\usepackage{multicol}
\usepackage{enumerate}
\usepackage[top=5em, bottom=5em, left=5em, right=5em]{geometry}
\usepackage{listings}
\usepackage{tikz}
\usetikzlibrary{positioning}

\setlength\parskip{1em}
\setlength\parindent{0em}

\title{Assignment 13}

\author{Hendrik Werner s4549775}

\begin{document}
\maketitle

This was done in collaboration with Constantin Blach (s4329872).

\section{} %1
\begin{enumerate}[(a)]
	\item %a
	It is proven that there exists a stable matching for any group size using the Gale–Shapley algorithm. We can just divide the people into two groups: Good and bad people. We then find a stable matching for both groups.

	Because a man eventually proposes to all women, everybody gets married in the end. Every woman that is proposed to will stick with the first proposer or switch him out later. In the end every woman is proposed to by some man so every woman is married which means that every man must also be married. (For every married woman there is a married man.)

	The marriages are stable because every man that prefers some woman over the one he gets at the end would have proposed to her earlier as she ranks higher than the woman he ended up with and got rejected.

	This means that there exists at least one stable matching where every good person is married to another good person and conversely ever bad person is married to another bad person.

	\item %b
\end{enumerate}

\section{} %2

\section{} %3

\section{} %4
\begin{enumerate}[(a)]
	\item %a
	\item %b
	\item %c
\end{enumerate}

\end{document}
