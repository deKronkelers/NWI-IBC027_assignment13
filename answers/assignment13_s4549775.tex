\documentclass[12pt, a4paper]{article}

\usepackage{amssymb}
\usepackage{multicol}
\usepackage{enumerate}
\usepackage[top=5em, bottom=5em, left=5em, right=5em]{geometry}
\usepackage{listings}
\usepackage{tikz}
\usetikzlibrary{positioning}

\setlength\parskip{1em}
\setlength\parindent{0em}

\title{Assignment 13}

\author{Hendrik Werner s4549775}

\begin{document}
\maketitle

This was done in collaboration with Constantin Blach (s4329872).

\section{} %1
\begin{enumerate}[(a)]
	\item %a
	It is proven that there exists a stable matching for any group size using the Gale–Shapley algorithm. We can just divide the people into two groups: Good and bad people. We then find a stable matching for both groups.

	Because a man eventually proposes to all women, everybody gets married in the end. Every woman that is proposed to will stick with the first proposer or switch him out later. In the end every woman is proposed to by some man so every woman is married which means that every man must also be married. (For every married woman there is a married man.)

	The marriages are stable because every man that prefers some woman over the one he gets at the end would have proposed to her earlier as she ranks higher than the woman he ended up with and got rejected.

	This means that there exists at least one stable matching where every good person is married to another good person and conversely ever bad person is married to another bad person.

	\item %b
	Every good person is ranked higher than every bad person. This means that every man will propose to all good women before he proposes to any bad woman. At the end every good woman is married to a good man because she rejects every bad man for any good man and all men propose to every good woman before proposing to any of the bad women. This means that at some point every good woman will be married to a good man because she will be proposed to by some good man (because he rather proposes to her than any of the bad women) at some point and reject any bad man she accepted earlier.

	After that there are only bad people left. We proved that there will exists a stable matching for the remaining people.

	In every stable matching all good people are married to another good person. This includes every good man being married to a good woman.
\end{enumerate}

\section{} %2

\section{} %3
The claim holds. An $n$ node simple graph $G$ is connected if every node has a degree of at least $n/2$.

Suppose we could split the graph into separate components that are not connected to each other. Each component has to contain at least $n/2 + 1$ nodes because we cannot connect to nodes outside of the component as that would break the component and each node has a degree of at least $n/2$. There are no self loops so the components needs at least $n/2$ other nodes any node in this component can connect to, which totals at least $n/2 + 1$.

We know the total number of nodes to be $n$. $\sum_{1 \leq k \leq c} (n/2 + 1) \leq n$, where $c$ is the number of components in the graph only holds for $c = 1$ so we know that there is only one component in the graph which means it is connected.

\section{} %4
\begin{enumerate}[(a)]
	\item %a
	Suppose $n = 2, M = 2$ and operating costs\\
	\begin{tabular}{|c||c|c|}
		\hline
		& Month 1 & Month 2\\
		\hline
		NY & 2 & 1\\
		\hline
		SF & 1 & 2\\
		\hline
	\end{tabular}

	The algorithm given produces $[SF, NY]$ with total operating costs of $4$.

	An optimal solution is $[NY, NY]$ with total operating costs of $3$.

	We have shown that the algorithm given does not produce plans with minimum cost.

	\item %b
	$n = 4, M = 0$ and operating costs\\
	\begin{tabular}{|c||c|c|c|c|}
		\hline
		& Month 1 & Month 2 & Month 3 & Month 4\\
		\hline
		NY & 1 & 0 & 1 & 0\\
		\hline
		SF & 0 & 1 & 0 & 1\\
		\hline
	\end{tabular}

	An optimal solution is $[SF, NY, SF, NY]$ which contains 3 moves.

	Every optimal plan must move at least 3 times because
	\begin{enumerate}
		\item there are 4 months,
		\item the moving cost is $M = 0$,
		\item and every month the cost is lower at another location compared to the previous month.
	\end{enumerate}

	\item %c
\end{enumerate}

\end{document}
